\documentclass[11pt]{article}

\usepackage[utf8x]{inputenc} % Включаем поддержку UTF8  
\usepackage[russian]{babel}
\title{RAM-disk driver}
\date{17.12.2016}
\author{Konstantin Butuzov}

\begin{document}
\pagenumbering{Arabic}
\begin{titlepage}
  \begin{center}
    \large
    МИНИСТЕРСТВО ОБРАЗОВАНИЯ И НАУКИ \\РОССИЙСКОЙ ФЕДЕРАЦИИ\\
    федеральное государственное автономное образовательное учреждение высшего образования\\
    «САНКТ-ПЕТЕРБУРГСКИЙ ГОСУДАРСТВЕННЫЙ УНИВЕРСИТЕТ 
    АЭРОКОСМИЧЕСКОГО ПРИБОРОСТРОЕНИЯ»

    Кафедра 43   
    \end{center}
	\vfill
	\noindent КУРСОВАЯ РАБОТА
    \normalsize{}\\
    \normalsize{ЗАЩИЩЕНА С ОЦЕНКОЙ}\\
    \normalsize{РУКОВОДИТЕЛЬ}
    
    \underline{ст.преп}
    \hspace{5cm}
    \underline{\hspace{3cm}}
    \hspace{3cm}
    \underline{М.Д.Поляк}
    \vfill
    
	\begin{center}
	\normalsize{ПОЯСНИТЕЛЬНАЯ ЗАПИСКА}\\
	\normalsize{К КУРСОВОЙ РАБОТЕ}\\
	\vfill
	\normalsize{Создание виртуального устройства}\\
	\vfill
    \textsc{по дисциплине: ОПЕРАЦИОННЫЕ СИСТЕМЫ И ОБОЛОЧКИ}\\
	\end{center}

	\vfill
	\noindent РАБОТУ ВЫПОЛНИЛ
	\normalsize{}\\  
	\normalsize{СТУДЕНТ ГР.}\hspace{1cm}\underline{4336}
	\hspace{2cm}
	\underline{\hspace{3cm}}
	\hspace{3cm}
	\underline{К.С.Бутузов}
\vfill

\begin{center}
  Санкт-Петербург, 2016 г.
\end{center}
\end{titlepage}
\newpage
\section{Цель работы}

Знакомство с устройством ядра ОС Linux. Получение опыта разработки драйвера устройства.
\section{Индивидуальное задание}

Создание виртуального устройства. Реализовать драйвер для создания RAM-диска.

Проверка правильности функционирования драйвера включает в себя создание RAM-диска, его форматирование в одну из популярных файловых систем, копирование файлов на диск и с диска, удаление RAM-диска.

\section{Сравнение с аналогами}

Как известно, RAM-диск – это программная технология, позволяющая хранить данные в быстродействующей оперативной памяти как на блочном устройстве. Обычно является составной частью операционной системы. В ряде случаев – это программа стороннего производителя.

Основные достоинства — высокая скорость чтения (измеряется гигабайтами в секунду), высокие показатели IOPS (операций ввода-вывода в секунду) — некоторые образцы оперативной памяти типа DDR3 позволяют достигать более 1 млн IOPS (у дисковых накопителей — 20—300 IOPS, NAND SSD — десятки—сотни тысяч IOPS), отсутствие дополнительных задержек при произвольном доступе, неограниченный ресурс перезаписи (в отличие от флеш-памяти). Среди недостатков — относительно малые ёмкости модулей оперативной памяти, потеря содержимого при отключении питания, высокая стоимость за гигабайт.

Linux реализует следующие виды RAM-дисков:

-специализированный архив в формате cpio для размещения модулей для начальной загрузки (initrd);

-файловая система, размещающаяся в памяти tmpfs (используется чаще всего для хранения временных данных, сохранение которых не актуально между перезагрузками и к которым нужен быстрый доступ);

-модуль brd, позволяющий создавать блочные устройства (вида /dev/ram0);

-модуль zram, позволяющий создавать блочные устройства вида /dev/zram0, хранящий данные в памяти в сжатом виде.

В задании требуется написать драйвер для создания RAM-диска, а также после его создания произвести с ним простейшие операции – его форматирование в одну из популярных файловых систем, копирование файлов на диск и с диска, удаление RAM-диска. Поэтому наиболее адекватным вариантов в нашем случае будет написать модуль brd.

Создание RAM-диска – достаточно стандартная и популярная операция, поэтому вариантов реализации в принципе немного, а отличаются они, как правило, размерами RAM-диска и секторов.

\section{Техническая документация}

\subsection{Сборка модуля}

Сначала необходимо собрать модуль ядра. Для этого написан Makefile. Сборку можно выполнить командой make. На некоторых устройствах нужно подробно прописывать путь, например: 
make -C/lib/modules/3.16.0-4-686-pae/build SUBDIRS=/home/konstant19961996/RAM-disk. Иногда нужно указать путь к исходникам ядра и к папке, где содержатся файлы

\subsection{Загрузка модуля в ядро}

Модуль загружается в ядро командой insmod dor.ko (где dor.ko - название модуля)

\subsection{Посмотреть структуру диска}

Проще всего посмотреть структуру диска при помощи команды lsblk

\subsection{Отформатировать раздел}

Отформатировать раздел можно с помощью команды mkfs.vfat /dev/rb1

\subsection{Смонтировать раздел}

Смонтировать отформатированный раздел можно командой mount -t vfat /dev/rb1 /mnt/

\subsection{Записать файл в раздел}

Записать файл в раздел можно при помощи команды cp. При этом стоит смотреть на размер файла, который вы хотите записать

\subsection{Размонтировать раздел}

Размонтировать раздел можно при помощи команды umount /mnt

\subsection{Выгрузить модуль}

Выгрузка модуля осуществляется командой rmmod

\newpage

\section{Выводы}

В ходе выполнения курсовой работы был написан драйвер для создания виртуального устройства, который позволяет ускорить работу с памятью в необходимых случаях. Для этого была изучена организация памяти в ОС Linux, поскольку данный драйвер – блочного устройства. Был изучен процесс написания модулей ядра под Linux, загрузка их в ядро, а также последующая работа с ними.

\newpage

\section{Приложение}
\subsection{Файл part.h}

\#ifndef PART\_H

\#define PART\_H

\#include <linux/types.h>

extern void copy\_mbr\_to\_br(u8 *disk);

\endif
\subsection{Файл part.c}

\#include <linux/string.h>

\#include "part.h"

\#define ARRAY\_SIZE(a) (sizeof(a) / sizeof(*a))

\#define SECTOR\_SIZE 512

\#define MBR\_SIZE SECTOR\_SIZE

\#define MBR\_DISK\_SIGNATURE\_OFFSET 440

\#define MBR\_DISK\_SIGNATURE\_SIZE 4

\#define PARTITION\_TABLE\_OFFSET 446

\#define PARTITION\_ENTRY\_SIZE 16 

\#define PARTITION\_TABLE\_SIZE 64 

\#define MBR\_SIGNATURE\_OFFSET 510

\#define MBR\_SIGNATURE\_SIZE 2

\#define MBR\_SIGNATURE 0xAA55

\#define BR\_SIZE SECTOR\_SIZE

\#define BR\_SIGNATURE\_OFFSET 510

\#define BR\_SIGNATURE\_SIZE 2

\#define BR\_SIGNATURE 0xAA55

typedef struct

\{
	unsigned char boottype;
	 
	unsigned char starthead;
	
	unsigned char startsec:6;
	
	unsigned char startcyl\_hi:2;
	
	unsigned char startcyl;
	
	unsigned char parttype;
	
	unsigned char endhead;
	
	unsigned char endsec:6;
	
	unsigned char endcyl\_hi:2;
	
	unsigned char endcyl;
	
	unsigned int abssecstart;
	
	unsigned int secpart;
	
\} PartEntry;

typedef PartEntry PartTable[4];

static PartTable def\_part\_table =

\{

	\{
	
		boottype: 0x00,
		
		starthead: 0x00,
		
		startsec: 0x2,
		
		startcyl: 0x00,
		
		parttype: 0x83,
		
		endhead: 0x00,
		
		endsec: 0x20,
		
		endcyl: 0x09,
		
		abssecstart: 0x00000001,
		
		secpart: 0x0000013F
		
	\},
	
	\{
	
		boottype: 0x00,
		
		starthead: 0x00,
		
		startsec: 0x1,
		
		startcyl: 0x0A, 
		
		parttype: 0x05,
		
		endhead: 0x00,
		
		endsec: 0x20,
		
		endcyl: 0x13,
		
		abssecstart: 0x00000140,
		
		secpart: 0x00000140
		
	\},
	
	\{
	
		boottype: 0x00,
		
		starthead: 0x00,
		
		startsec: 0x1,
		
		startcyl: 0x14,
		
		parttype: 0x83,
		
		endhead: 0x00,
		
		endsec: 0x20,
		
		endcyl: 0x1F,
		
		abssecstart: 0x00000280,
		
		secpart: 0x00000180
		
	\},
	
	\{
	
	\}
	
\};

static unsigned int def\_log\_part\_br\_cyl[] = \{0x0A, 0x0E, 0x12\};

static const PartTable def\_log\_part\_table[] =

\{

	\{
	
		\{
		
			boottype: 0x00,
			
			starthead: 0x00,
			
			startsec: 0x2,
			
			startcyl: 0x0A,
			
			parttype: 0x83,
			
			endhead: 0x00,
			
			endsec: 0x20,
			
			endcyl: 0x0D,
			
			abssecstart: 0x00000001,
			
			secpart: 0x0000007F
			
		\},
		
		\{
		
			boottype: 0x00,
			
			starthead: 0x00,
			
			startsec: 0x1,
			
			startcyl: 0x0E,
			
			parttype: 0x05,
			
			endhead: 0x00,
			
			endsec: 0x20,
			
			endcyl: 0x11,
			
			abssecstart: 0x00000080,
			
			secpart: 0x00000080
			
		\},
		
	\},
	
	\{
	
		\{
		
			boottype: 0x00,
			
			starthead: 0x00,
			
			startsec: 0x2,
			
			startcyl: 0x0E,
			
			parttype: 0x83,
			
			endhead: 0x00,
			
			endsec: 0x20,
			
			endcyl: 0x11,
			
			abssecstart: 0x00000001,
			
			secpart: 0x0000007F
			
		\},
		
		\{
		
			boottype: 0x00,
			
			starthead: 0x00,
			
			startsec: 0x1,
			
			startcyl: 0x12,
			
			parttype: 0x05,
			
			endhead: 0x00,
			
			endsec: 0x20,
			
			endcyl: 0x13,
			
			abssecstart: 0x00000100,
			
			secpart: 0x00000040
			
		\},
		
	\},
	
	\{
	
		\{
		
			boottype: 0x00,
			
			starthead: 0x00,
			
			startsec: 0x2,
			
			startcyl: 0x12,
			
			parttype: 0x83,
			
			endhead: 0x00,
			
			endsec: 0x20,
			
			endcyl: 0x13,
			
			abssecstart: 0x00000001,
			
			secpart: 0x0000003F
			
		\},
		
	\}
	
\};


static void copymbr(u8 *disk)

\{

	memset(disk, 0x0, MBR\_SIZE);
	
	*(unsigned long *)(disk + MBR\_DISK\_SIGNATURE\_OFFSET) = 0x36E5756D;
	
	memcpy(disk + PARTITION\_TABLE\_OFFSET, &def\_part\_table, PARTITION\_TABLE\_SIZE);
	
	*(unsigned short *)(disk + MBR\_SIGNATURE\_OFFSET) = MBR\_SIGNATURE;
	
\}

static void copybr(u8 *disk, int start\_cylinder, const PartTable *part\_table)

\{

	disk += (start\_cylinder * 32  * SECTOR\_SIZE);
	
	memset(disk, 0x0, BR\_SIZE);
	
	memcpy(disk + PARTITION\_TABLE\_OFFSET, part\_table,
		PARTITION\_TABLE\_SIZE);
		
	*(unsigned short *)(disk + BR\_SIGNATURE\_OFFSET) = BR\_SIGNATURE;
	
\}

void copy\_mbr\_ti\_br(u8 *disk)

\{
	int i;

	copymbr(disk);
	
	for (i = 0; i < ARRAY\_SIZE(def\_log\_part\_table); i++)
	
	\{
	
		copybr(disk, def\_log\_part\_br\_cyl[i], &def\_log\_part\_table[i]);
		
	\}
	
\}

\subsection{Файл ramdevice.h}

\#ifndef RAMDEVICE\_H

\#define RAMDEVICE\_H

\#define RB\_SECTOR\_SIZE 512

extern int ramdevice\_initial(void);

extern void ramdevice\_clean(void);

extern void ramdevice\_write(sector\_t sector\_off, u8 *buffer, unsigned int sectors);

extern void ramdevice\_read(sector\_t sector\_off, u8 *buffer, unsigned int sectors);

\#endif

\subsection{Файл ramdevice.c}

\#include <linux/types.h>

\#include <linux/vmalloc.h>

\#include <linux/string.h>

\#include <linux/errno.h>

\#include "ramdevice.h"

\#include "part.h"

\#define RB\_DEVICE\_SIZE 1024 

static u8 *dev\_data;

int ramdevice\_initial(void)

\{

	dev\_data = vmalloc(RB\_DEVICE\_SIZE * RB\_SECTOR\_SIZE);
	
	if (dev\_data == NULL)
	
		return -ENOMEM;
	copy\_mbr\_to\_br(dev\_data);
	
	return RB\_DEVICE\_SIZE;
	
\}

void ramdevice\_clean(void)

\{

	vfree(dev\_data);
	
\}

void ramdevice\_write(sector\_t sector\_off, u8 *buffer, unsigned int sectors)

\{

	memcpy(dev\_data + sector\_off * RB\_SECTOR\_SIZE, buffer,
		sectors * RB\_SECTOR\_SIZE);
		
\}

void ramdevice\_read(sector\_t sector\_off, u8 *buffer, unsigned int sectors)

\{

	memcpy(buffer, dev\_data + sector\_off * RB\_SECTOR\_SIZE,
		sectors * RB\_SECTOR\_SIZE);
		
\}

\subsection{Файл ramblock.c}

\#include <linux/module.h>

\#include <linux/kernel.h>

\#include <linux/version.h>

\#include <linux/fs.h>

\#include <linux/types.h>

\#include <linux/spinlock.h>

\#include <linux/genhd.h> 

\#include <linux/blkdev.h> 

\#include <linux/hdreg.h> 

\#include <linux/errno.h>


\#include "ramdevice.h"

\#define RB\_FIRST\_MINOR 0

\#define RB\_MINOR\_CNT 16

static u\_int rb\_major = 0;


static struct rbdevice

\{

	unsigned int size;
	
	spinlock\_t lock;
	
	struct request\_queue *rb\_queue;
	
	struct gendisk *rb\_disk;
	
\} rb\_dev;

static int rbopen(struct block\_device *bdev, fmode\_t mode)

\{

	unsigned unit = iminor(bdev->bd\_inode);

	printk(KERN\_INFO "rb: Device is opened\n");
	
	printk(KERN\_INFO "rb: Inode number is \%d\n", unit);

	if (unit > RB\_MINOR\_CNT)
	
		return -ENODEV;
		
	return 0;
	
\}

\#if (LINUX\_VERSION\_CODE < KERNEL\_VERSION(3,10,0))

static int rbclose(struct gendisk *disk, fmode\_t mode)

\{

	printk(KERN\_INFO "rb: Device is closed\n");
	
	return 0;
	
\}

\#else

static void rbclose(struct gendisk *disk, fmode\_t mode)

\{

	printk(KERN\_INFO "rb: Device is closed\n");
	
\}

\#endif

static int rbgetgeo(struct block\_device *bdev, struct hd\_geometry *geo)

\{

	geo->heads = 1;
	
	geo->cylinders = 32;
	
	geo->sectors = 32;
	
	geo->start = 0;
	
	return 0;
	
\}


static int rbtransfer(struct request *req)

\{

	int dir = rq\_data\_dir(req);
	
	sector\_t start\_sector = blk\_rq\_pos(req);
	
	unsigned int sector\_cnt = blk\_rq\_sectors(req);

\#if (LINUX\_VERSION\_CODE < KERNEL\_VERSION(3,14,0))

\#define BV\_PAGE(bv) ((bv)->bv\_page)

\#define BV\_OFFSET(bv) ((bv)->bv\_offset)

\#define BV\_LEN(bv) ((bv)->bv\_len)

	struct bio\_vec *bv;
	
\#else

\#define BV\_PAGE(bv) ((bv).bv\_page)

\#define BV\_OFFSET(bv) ((bv).bv\_offset)

\#define BV\_LEN(bv) ((bv).bv\_len)

	struct bio\_vec bv;
	
\#endif

	struct req\_iterator iter;

	sector\_t sector\_offset;
	
	unsigned int sectors;
	
	u8 *buffer;

	int ret = 0;


	sector\_offset = 0;
	
	rq\_for\_each\_segment(bv, req, iter)
	
	\{
	
		buffer = page\_address(BV\_PAGE(bv)) + BV\_OFFSET(bv);
		
		if (BV\_LEN(bv) \% RB\_SECTOR\_SIZE != 0)
		
		\{
		
			printk(KERN\_ERR "rb: Should never happen: "
			
				"bio size (\%d) is not a multiple of RB\_SECTOR\_SIZE (\%d).\n"
				
				"This may lead to data truncation.\n",
				
				BV\_LEN(bv), RB\_SECTOR\_SIZE);
				
			ret = -EIO;
			
		\}
		
		sectors = BV\_LEN(bv) / RB\_SECTOR\_SIZE;
		
		printk(KERN\_DEBUG "rb: Start Sector: \%llu, Sector Offset: \%llu; Buffer: \%p; Length: \%u sectors\n",
		
			(unsigned long long)(start\_sector), (unsigned long long)(sector\_offset), buffer, sectors);
			
		if (dir == WRITE) 
		
		\{
		
			ramdevice\_write(start\_sector + sector\_offset, buffer, sectors);
			
		\}
		
		else 
		
		\{
		
			ramdevice\_read(start\_sector + sector\_offset, buffer, sectors);
			
		\}
		
		sector\_offset += sectors;
		
	\}
	
	if (sector\_offset != sector\_cnt)
	
	\{
	
		printk(KERN\_ERR "rb: bio info doesn't match with the request info");
		
		ret = -EIO;
		
	\}

	return ret;
	
\}
	

static void rbrequest(struct request\_queue *q)

\{

	struct request *req;
	
	int ret;

	while ((req = blk\_fetch\_request(q)) != NULL)
	
	\{
	
\#if 0
		
		if (!blk\_fs\_request(req))
		
		\{
		
			printk(KERN\_NOTICE "rb: Skip non-fs request\n");
			
			\_\_blk\_end\_request\_all(req, 0);
			
			continue;
			
		\}
		
\#endif

		ret = rb\_transfer(req);
		
		\_\_blk\_end\_request\_all(req, ret);
		
	\}
	
\}


static struct block\_device\_operations rb\_fops =

\{
	
	.owner = THIS\_MODULE,
	
	.open = rb\_open,
	
	.release = rb\_close,
	
	.getgeo = rb\_getgeo,
	
\};
	

static int \_\_init rbinit(void)

\{

	int ret;

	if ((ret = ramdevice\_init()) < 0)
	
	\{
	
		return ret;
		
	\}
	
	rb\_dev.size = ret;

	rb\_major = register\_blkdev(rb\_major, "rb");
	
	if (rb\_major <= 0)
	
	\{
	
		printk(KERN\_ERR "rb: Unable to get Major Number\n");
		
		ramdevice\_cleanup();
		
		return -EBUSY;
		
	\}

	spin\_lock\_init(&rb\_dev.lock);
	
	rb\_dev.rb\_queue = blk\_init\_queue(rb\_request, &rb\_dev.lock);
	
	if (rb\_dev.rb\_queue == NULL)
	
	\{
	
		printk(KERN\_ERR "rb: blk\_init\_queue failure\n");
		
		unregister\_blkdev(rb\_major, "rb");
		
		ramdevice\_cleanup();
		
		return -ENOMEM;
		
	\}
	
	rb\_dev.rb\_disk = alloc\_disk(RB\_MINOR\_CNT)
	;
	if (!rb\_dev.rb\_disk)
	
	\{
	
		printk(KERN\_ERR "rb: alloc\_disk failure\n");
		
		blk\_cleanup\_queue(rb\_dev.rb\_queue);
		
		unregister\_blkdev(rb\_major, "rb");
		
		ramdevice\_cleanup();
		
		return -ENOMEM;
		
	\}

	rb\_dev.rb\_disk->major = rb\_major;
	
	rb\_dev.rb\_disk->first\_minor = RB\_FIRST\_MINOR;
	
	rb\_dev.rb\_disk->fops = &rb\_fops;
	
	rb\_dev.rb\_disk->private\_data = &rb\_dev;
	
	rb\_dev.rb\_disk->queue = rb\_dev.rb\_queue;
	
	sprintf(rb\_dev.rb\_disk->disk\_name, "rb");
	
	set\_capacity(rb\_dev.rb\_disk, rb\_dev.size);
	
	add\_disk(rb\_dev.rb\_disk);
	
	printk(KERN\_INFO "rb: Ram Block driver initialised (\%d sectors; \%d bytes)\n",
		rb\_dev.size, rb\_dev.size * RB\_SECTOR\_SIZE);

	return 0;
	
\}

static void \_\_exit rbcleanup(void)

\{

	del\_gendisk(rb\_dev.rb\_disk);
	
	put\_disk(rb\_dev.rb\_disk);
	
	blk\_cleanup\_queue(rb\_dev.rb\_queue);
	
	unregister\_blkdev(rb\_major, "rb");
	
	ramdevice\_cleanup();
	
\}

module\_init(rb\_init);

module\_exit(rb\_cleanup);

MODULE\_LICENSE("GPL");

MODULE\_AUTHOR("Konstantin Butuzov <bks1906@gmail.com>");

MODULE\_DESCRIPTION("Ram Block Driver");

MODULE\_ALIAS\_BLOCKDEV\_MAJOR(rb\_major);

\subsection{Makefile}

ifeq (\$(KERNELRELEASE),)

	KERNEL\_SOURCE := /lib/modules/\$(shell uname -r)/build
	
	PWD := \$(shell pwd)

module:

	\$(MAKE) -C \$(KERNEL\_SOURCE) SUBDIRS=\$(PWD) modules

clean:

	\$(MAKE) -C \$(KERNEL\_SOURCE) SUBDIRS=\$(PWD) clean


else

	obj-m := dor.o
	
	dor-y := ramblock.o ramdevice.o part.o

endif

\end{document}